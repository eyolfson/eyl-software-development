\documentclass{article}

\usepackage{fontspec}
\setsansfont{Overpass}[Scale=MatchLowercase]
\setmonofont{Overpass Mono}[Scale=MatchLowercase]

\usepackage[letterpaper, margin=72pt]{geometry}

\newcommand{\binary}[1]{$(\texttt{#1})_{2}$}
\newcommand{\hexadecimal}[1]{$(\texttt{#1})_{16}$}

\title{Teensy 3.2 Boot Process}

\begin{document}

\maketitle

\section{Watchdog}

The write to \texttt{WDOG\_STCTRLH} allows updates.
I don't think anything uses the watchdog.

\section{Real time clock}

Doesn't appear to be used.

\section{Multipurpose clock generator (MCG)}

Sets up the clock source, likely important.

\section{System integration module}

System control, likely important.

\section{Onboard LED}

Schematic shows it on \texttt{PTC5}.
It's connected to the Teensy as pin 13.

First, configure pin 13 for output.
Write \hexadecimal{00000144} to \texttt{PORTC\_PCR5} at \hexadecimal{4004B014}.
Bits 8, 6, and 2 are 1.
The \texttt{MUX} field is \binary{001}, meaning the pin is a GPIO pin.
Bit 6 means high drive strength enable (whatever that means).
Bit 2 means slow slew rate (again, whatever that means).

Afterwards write high or low to pin 13.
The example code writes high.
Write \hexadecimal{00000003} to \texttt{PORTC\_PCR5} at \hexadecimal{4004B014}.

\section{Resources}

\texttt{MK20DX256VLH7} reference manual.

\end{document}
